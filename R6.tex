\documentclass{article}
\usepackage{booktabs}
\usepackage[hidelinks]{hyperref}
\usepackage[margin=1in]{geometry}
\usepackage{arydshln}
\begin{document}

\begin{table}
\centering
\caption{Response to PNAS Editing Proof Comments (\textbf{pnas.2306412121})}
\begin{tabular}{p{0.5in}p{6in}}
\toprule
Q1 & The address should be written with the zip code at the end per typical US address convention.
a Yale School of Management, Yale University, New Haven, CT 06511; 
b John Heinz III College of Public Policy and Management, Carnegie Mellon University,  Pittsburgh, PA 15213;
and 
c Department of Sociology, College of Liberal Arts, Purdue University, West Lafayette, IN 47907\\
\midrule
Q2 & OK. \\
\midrule
Q3 & OK.\\
\midrule
Q4 & OK.\\
\midrule
Q5 & No division information for David Krackhardt and Vineet Kumar.
%For David Krackhardt, please replace "Heinz College" with "H. John Heinz III College of Public Policy and Management"
\\
\midrule
Q6 & Regarding contribution, there is some inconsistency. First, we don't use reagents so I have removed that. In this academic field, it is typically called ``empirical analysis'' so we would prefer to use that terminology and hope that would be ok. We request that you please replace the text with the following: ``V.K., D.K., and S.F. designed research; V.K., D.K., and S.F. performed research; V.K. contributed new theoretical results; V.K. performed empirical analysis on collected data; and V.K. wrote the paper.''\\
\midrule
Q7 & Yes\\
\midrule
Q8 & No change to SI.\\
\midrule
Q9 & We are ok with you removing the italics. Otherwise, we request that you provide a Latex file, and we can remove it. Since edits have been made by your team after our submission, we no longer have the latest version. We would very much appreciate your guidance here.\\
\midrule
Q10 & OK.\\
\midrule
Q11 & This is the only location where we would like to retain the claim to novelty, and we are perfectly fine if such claims are removed from everywhere else in the text, as you have suggested. The reason we would want it here is that the intervention strategy is quite different, but appears superficially similar to prior work. We have directly shown how the mathematical properties are substantially different, so to avoid any potential confusion for the reader, we think it might be helpful to retain the claim to novelty here. Thank you for your consideration.\\
\midrule
Q12 & MIT: Massachusetts Institute of Technology, 
CMU: Carnegie Mellon University, 
UT: University of Texas, 
UC: University of California\\
\midrule
Q13 & (8) Book Chapter:
Book Title: Networks in Marketing, 
Editor: Dawn Iacobucci, 
Chapter Title: Structural leverage in marketing, 
Author: David Krackhardt, 
Pages: 50--59, 
Year: 1996, 
Publisher: Sage, Thousand Oaks, CA  \\
\midrule
Q14 &
(Reference 9) R. Czaja, “Sampling with probability proportional to size” in Encyclopedia of Biostatistics (2005),
vol. 7.
\textbf{Editors:} Peter Armitage and Theodore Colton, 
\textbf{Publisher:} John Wiley \& Sons, Ltd 
URL: \url{https://onlinelibrary.wiley.com/page/book/10.1002/0470011815/homepage/editorscontributors.html} \\
Q14 & (Reference 18) D. Kempe, J. Kleinberg, É. Tardos, ``Maximizing the spread of influence through a social network'' in Proceedings of the Ninth ACM SIGKDD International Conference On Knowledge Discovery and Data
Mining (2003), pp. 137–146. 
\textbf{Editors:} Lise Getoor, Ted Senator, Pedro Domingos, Christos Faloutsos, 
\textbf{Publisher:} Association for Computing Machinery, New York, NY, United States \\
Q14 & (Reference 25) S. Torabi, K. Beznosov, “Privacy aspects of health related information sharing in online social networks” in 2013 USENIX Workshop on Health Information Technologies (2013).
\textbf{Editors:} Kevin Fu, Darren Lacey, Zachary Peterson, \textbf{Publisher:} {USENIX} Association, Berkeley, CA\\

\bottomrule
\end{tabular}
\end{table}

\end{document}
